\documentclass{article}

\usepackage{graphicx}
\usepackage{geometry}
\usepackage{enumitem}

\geometry{a4paper, margin=1in}

\begin{document}
	
	
	\section{ESP-Home}
	\subsection{Introduction to ESP-Home}
	
	Modern life is now completely dependent on smart home automation, which provides ease, improved security, and energy efficiency. Within the domain of open-source smart home systems, ESPHome is a platform designed to simplify and optimize do-it-yourself home automation projects. ESPHome, which was first developed as a component of Home Assistant, is now well-known for its user-friendly interface and seamless device connectivity. The ESP-32 device could be flashed using a variety of methods, including Platform-IO, ESP-Home, Arduino-IDE, etc. Consider your application and the problem you wish to address before choosing which platform to use.
	
	\begin{figure}[h]
		\centering
		\includegraphics[width=0.6\textwidth]{"C:/Users/axitk/OneDrive/Desktop/THM Document/IOT Project/Report/Image/ESPHome"}
		\caption{ESPHome}
		\small Source: https://esphome.io/
	\end{figure}
	
	\subsubsection{Advantages of using ESP-Home:}
	\begin{enumerate}
		\item ESPHome is intended to integrate easily with the ESP8266 and ESP32 microcontrollers, which are popular among makers because of their low cost, low power consumption, and integrated Wi-Fi.
		\item ESPHome uses YAML configuration files, which is one of its unique characteristics. They offer a great level of flexibility and customization by defining devices, automation rules, and integrations.
		\item ESPHome is compatible with several protocols for communication, such as MQTT and API-based exchanges. This allows devices and platforms from various smart homes to work together.
		\item By enabling users to configure sensors, actuators, and other components using an easy-to-understand configuration language, ESPHome streamlines the device management process.
		\item ESPHome easily interfaces with the well-liked open-source home automation platform as a part of the Home Assistant ecosystem. Through the provision of a single interface for controlling all linked devices, this integration improves the entire smart home experience.
	\end{enumerate}
	
	\subsubsection{Getting Started with ESPHome}
	If you want to create custom firmware for your ESP8266/ESP32 devices, ESPHome is also the possible answer. We'll walk over how to quickly and easily set up a basic "node". The ESPHome may be used in a variety of ways. Some of the services are as follows:
	\begin{enumerate}[label=(\arabic*)]
		\item From Home Assistant
		\item Using the command line
		\item Migrating from Tasmota
		\item Using the Docker Container
	\end{enumerate}
	
	Some of the prerequisites you need to have before getting started with this method:
	\begin{enumerate}[label=(\arabic*)]
		\item You should be familiar with the Docker container and how it is working.
		\item Portioner should be running which is used as a container management tool.
		\item You should be familiarised with the YAML file of the configuration.
	\end{enumerate}
	
	In this project, we are using the Method using the Docker container. If you want to start with the ESP-Home via Docker container then you need to follow these steps.
	
	\subsubsection{Installation of ESP-Home Via Command Line}
	You need to follow the below step for installing the ESP-Home if you are using the method via the command line.
	\begin{enumerate}
		\item Open a Docker Container App
		\item Create an image of the Portioner container management tool we used in this project. Follow this reference for more detail Source: \\{https://docs.portainer.io/start/install-ce/server/docker/linux}
		\item Open a portainer container by typing on the browser (preferred Google Chrome). \\Command: {https://mt-labor.iem.thm.de:9443}
		\item Create a stack using the YAML File of ESP-Home as follows and upload in the portioner. 
		
		\begin{verbatim}
			services:
			esphome:
			image: esphome/esphome
			container_name: esphome
			restart: unless-stopped
			ports:
			- "20000:6052" # Web dashboard
			networks:
			- dev-net
			volumes:
			- team4-esphome_team4_esphome_conf:/config
			environment:
			USERNAME: "admin"
			PASSWORD: "iotlab2023team4"
			ESPHOME_DASHBOARD_USE_PING: true
			
			volumes:
			team4-esphome_team4_esphome_conf:
			
			networks:
			dev-net:
		\end{verbatim}
	\end{enumerate}
	
	
	\begin{enumerate}[resume]
		\item Open the ESP-Home dashboard by typing below in the web browser(preferred Google Chrome). \\Command: {https://mt-labor.iem.thm.de:6052}
		\item After opening the site, you will see the below screen as a dashboard. This is the initial setup of the ESP-Home.
	\end{enumerate}
	
	\begin{figure}[h]
		\centering
		\includegraphics[width=0.6\textwidth]{"C:/Users/axitk/OneDrive/Desktop/THM Document/IOT Project/Report/Image/ESPHome dashboard"}
		\caption{ESPHome Dashboard}
	\end{figure}
	
	Please refer to the port detail in the GitLab for more details regarding the configured port.
	
\end{document}
