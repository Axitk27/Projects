The power system for the Mecanum wheeled car is designed around two \ac{lipo} batteries connected in series. Each \ac{lipo} cell has a nominal voltage of \num{3.7}~\si{\volt}, resulting in a total voltage of \num{7.4}~\si{\volt} when combined. This configuration provides sufficient voltage to drive the motors and other key components of the car, offering a compact and efficient energy solution for the system. Initially, the plan was to use a dedicated \num{5}~\si{\volt} output step-down converter to supply power to the ESP32 microcontroller. However, this approach was abandoned due to issues arising from the converter's inability to handle the low power requirements of the ESP32 in isolation. The step-down converter was designed for heavier loads, and with only the microcontroller connected, it was operating under a "too light" load condition, causing instability in the power supply.\\
To resolve this, the power system was simplified. The \num{7.4}~\si{\volt} output from the \ac{lipo} batteries became the primary supply voltage for both the motor driver and the ESP32 microcontroller. This configuration is advantageous for two reasons. The motor driver is capable of handling the \num{7.4}~\si{\volt}, which is adequate to power the motors at optimal performance. The ESP32 development board has its own built-in voltage regulator capable of stepping down input voltages as high as \num{12}~\si{\volt} to its required \num{3.3}~\si{\volt} operating voltage. Therefore, the same \num{7.4}~\si{\volt} battery supply could be used to power the ESP32 directly, eliminating the need for an external converter and simplifying the system’s overall design. The ESP32 development board includes a dedicated pin that outputs a regulated \num{3.3}~\si{\volt}, enabling the powering of additional components, such as sensors, that lack their own voltage regulation. This simplifies integration by providing a stable power source for peripherals directly from the microcontroller. This revised power strategy enhances the robustness of the power system while reducing component count and complexity, ensuring reliable operation of both the microcontroller and motors.