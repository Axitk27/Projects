
	{\Large \textbf{Mecanum Car Commands}}\\
	Use Wi-Fi to establish a connection between the Mecanum Car and a client device.\\
	\begin{tcolorbox}[width=\linewidth, sharp corners=all, colback=white!95!black, boxrule=0pt]
		{\large \textbf{Send Command \& Receive Response}}
	\end{tcolorbox}
	Mecanum Car IP: \textbf{192.168.10.1} UDP Command Port: \textbf{4444} UDP Status Port: \textbf{4445}
	
	\begin{tabular}{R{0.1\textwidth} L{0.85\textwidth}}
		Step 1:\newline \phantom{-} & Set up a UDP client on the client device to send and receive messages from the Mecanum Car via the same port. \\
		Step 2:\newline \phantom{-} & Send any available command to the Mecanum Car to control it. \\
	\end{tabular}
	
	\begin{tcolorbox}[width=\linewidth, sharp corners=all, colback=white!95!black, boxrule=0pt]
		{\large \textbf{Control Commands}}
	\end{tcolorbox}
	\begin{tabularx}{\textwidth}{c|c|c}
		\hline
		\textbf{Command} & \textbf{Description} & \textbf{Possible Response} \\
		\hline
		rc x y z & \makecell{Apply speed to  x y or z direction.\\x (backward - forward) = -127 to 127\\y (left - right) = -127 to 127\\z (rotation ccw - cw) = -127 to 127} & -/error \\
		\hline
		keepalive & \makecell{Keeps connection to Mecanum Car alive.\\Prevents timeout without changing last command.} & ok/error \\
		\hline
		stop & \makecell{Set all directions to 0.} & ok/error \\
		\hline
	\end{tabularx}
	\\
	\begin{tcolorbox}[width=\linewidth, sharp corners=all, colback=white!95!black, boxrule=0pt]
		{\large \textbf{Status Message | Data Type: String}}
	\end{tcolorbox}
	\textbf{Data string received every second while controlling via WiFi}\\
	"batV:\%.2f,batP:\%d,mA:\%d,mB:\%d,mC:\%d,mD:\%d,:\textbackslash n"\\
	\\
	\textbf{Description}\\
	"batV" = Measured voltage of battery in Volt\\
	"batP" = Charge of battery in Percent\\
	"mX" = Duty cycle of motor X in Percent\\
